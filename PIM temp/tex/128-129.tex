\documentclass[twoside]{article}
\usepackage[russian]{babel}

%\usepackage[utf8x]{inputenc}
%\usepackage[font=small,labelfont=bf]{caption}
%\usepackage[a5paper, left=1cm, right=1cm, top=1cm, bottom=1cm]{geometry}
\usepackage[paperheight=16cm,paperwidth=12cm,textwidth=10cm]{geometry}

\usepackage{graphicx}
\usepackage{mathrsfs}
%\usepackage[cp1251]{inputenc}
\usepackage{fancyhdr}


\pagestyle{fancy}
\renewcommand{\headrulewidth}{0.4pt}

\cfoot{}
\newcommand{\qedsymbol}
{
  \rule{2.5mm}{2.5mm}
}

\newcounter{nrazd}
\setcounter{nrazd}{8}
\newcommand{\razd}[1]
{
  \addtocounter{nrazd}{1} 
  \setcounter{equation}{0} 
  \textbf{\thenrazd. #1} 
}

%\newcounter{npara}
%\setcounter{npara}{9}
%\newcommand{\para}[1]
%{
%  \addtocounter{npara}{1}
%  \setcounter{equation}{0} 
%  \textbf{\thenrazd. #1} 
%}

\renewcommand{\theequation}
{
  \thenrazd.\arabic{equation}
}

\newcommand{\bm}[1]
{
  \mbox{\boldmath{\(#1\)}}
}

\setcounter{page}{128}




\begin{document}

  \fancyhead[LE, RO]{\thepage}
  \fancyhead[C]{\headnotewidth ГЛ. III. ОСНОВНЫЕ ГРУППЫ КОНКРЕТНЫХ СИСТЕМ УРАВНЕНИЙ}

    \noindent Это решение дает по формулам (7.5) алгебру Ли $L^3$ операторов вида (4.1), допускаемых в данном случае уравнением (7.1), базис которой таков:
    \setcounter{equation}{2}
    \begin{equation} \label{formula8.3}
      \zeta_1 = (0,1,0),\;
      \zeta_2 = (x,2y,0),\;
      \zeta_3 = (-4xy,-4y^2,(x^2+2y)z)
    \end{equation}

  \razd{Классификационный результат.}Окончательный результат групповой классификации параболических нормальных форм (7.1) уравнения второго порядка формулируется в следующей теореме.

    Т\;е\;о\;р\;е\;м\;а. \textit{Уравнение} (7.1) \textit{допускает какой-либо оператор} (4.1)\textit{, если и только если оно равносильно уравнению этого вида с функцией $H$ , не зависящей от одной из координат, $x$ или $y$. Расширение основной алгебры Ли возможно, только если уравнение} (7.1) \textit{равносильно такому же уравнению с функцией} $ H=mx^{-2} $ \textit{(m=const); последнее при} $m \neq 0 $ \textit{допускает алгебру Ли} $ L^3$\textit{операторов} (\ref{formula8.3})\textit{, а при} $m=0$ \textit{--- алгебру Ли} $L^5$\textit{, найденную в} 6.8.

    Д\;о\;к\;а\;з\;а\;т\;е\;л\;ь\;с\;т\;в\;о.  В силу предыдущего надо доказать только первую часть утверждения. Легко проверить, пользуясь формулой подобия векторных полей 1.16, что в случае преобразования, определяемого формулами (7.2), вектор ($\xi, \vartheta, \eta, \sigma$) переходит в вектор ($\overline{\xi},\overline{\eta}, \overline{\sigma}, \overline{z}$) с координатами
    \[
      \overline{\xi}=a\xi+(a\prime x+b\prime)\eta,\;
      \overline{\eta}=a^2\eta,\;
      \overline{\sigma}=\sigma+\rho_x\xi+\rho_y\eta,
    \]
    \noindent где $\rho = \frac{a^\prime}{4a}x^2+\frac{b^\prime}{2a}x+c$. В силу равенств (7.5) это влечет преобразование функций $\varphi, \psi, \chi$ по формулам
    
    \begin{equation}\label{formula9.1}
      \overline{\varphi}=a^2\varphi,\;
      \overline{\psi}=a\psi+4b^\prime\varphi,\;
      \overline{\chi}=\chi-\frac{b^\prime}{2a}\psi-4c^\prime\varphi.
    \end{equation}

    Пусть уравнение (7.1) допускает ненулевой оператор (4.1) с функциями $\varphi, \psi, \chi$. Если $\varphi \neq 0$, то можно выбрать функции $a, b, c$ такими, что формулы (\ref{formula9.1}) дадут $\overline{\varphi}=$ const $\neq 0$, $\overline{\psi}=\overline{\chi}=0$. В этом случае определяющее уравнение (7.6) после преобразования (7.2), (7.3) примет вид $4\overline{\varphi} \partial_{\overline{y}} \overline{H}$, откуда следует, что $\overline{H}$ не зависит от $\overline{y}$. Если же $\varphi=0 $, но $\psi \neq 0 $, то с помощью (\ref{formula9.1}) можно сделать $\overline{\psi}$ = const $\neq 0$ и $\overline{\chi}=0$ определяющее уравнение (7.6) после преобразования (7.2), (7.3) будет $\overline{\psi}\partial\overline{x}\overline{H}=0 $, т. е. $\overline{H}$ не зависит от x. Наконец, из (7.6) следует, что если $\varphi=\psi=0$, то оператор (4.1) нулевой.
    \qedsymbol
    \par

    
    %\fancyhead[R]{\thepage}
    %\fancyhead[C]{\footnotesize ГЛ. III ОСНОВНЫЕ ГРУППЫ КОНКРЕТНЫХ СИСТЕМ УРАВНЕНИЙ}
    С\;л\;е\;д\;с\;т\;в\;и\;е. \textit{Если уравнение} (7.1) \textit{ два линейно независимых оператора вида} (4.1)\textit{, то оно равносильно уравнению} 
    $$z_{11} = z_2+mx^{-2}z.$$

\newpage


  \setcounter{nrazd}{0}
  \fancyhead[C]{\footnotesize \S 10. УРАВНЕНИЯ ПОГРАНИЧНОГО СЛОЯ}
  %\fancyhead[R]{\thepage}


  %\paragraph{\textsection 10. Уравнения пограничного слоя}

  \begin{center}
    \textbf{\textsection 10. Уравнения пограничного слоя}
  \end{center}\noindent 

  \razd{Описание системы уравнений.} Рассматриваются известные из гидродинамики уравнения двумерного нестационарного пограничного слоя (уравнения Прандтля), описывающие движения вязкой несжимаемой жидкости вблизи непроницаемой твердой поверхности при больших числах Рейнольдса. Эта система имеет тип \textit{E}(3,3,2,3) и будет описываться в индивидуальных обозначениях координат, прилитых и гидродинамике. Здесь $\bm{X}=\bm{R^3} (t, x, y)$, $\bm{Y}=\bm{R^3}(u, v, p)$, причем $t$ --- время, $x, y$ --- пространственные координаты, $u, v$ — координаты вектора скорости и $p$ --- давление. Оператор дифференцирования отображений $\bm{Z}=\bm{X}\times \bm{Y}\rightarrow \bm{P}$ записывается в виде $\partial = (\partial_t,\; \partial_x,\; \partial_y,\; \partial_u,\; \partial_v,\; \partial_p)$ и для частных производных используются сокращенные обозначения с соответствующими индексами, например $\partial_tu = u_t,\; \partial_xu = u_x,\; \partial_y^2u = u_{yy}$ и т. п.

    Не нарушая общности рассмотрения, можно принять, что плотность жидкости и коэффициент вязкости равны единице. При этих соглашениях исследуемая система уравнений такова:

    \begin{equation}\label{formula1.1}
      \left.
      \begin{array}{rl}
        u_t+uu_x+vu_y+p_x  =  u_{yy}, \\
        p_y  =  0, \\
        u_x+v_y=0.\\
      \end{array}
      \right\}
    \end{equation}

    Давление $p$ играет в этой системе особую роль. В силу второго уравнения $p$ есть функция только от $(t, x)$. В задачах гидродинамики эта функция часто может считаться заданной. В этом случае, если функцию $p_x=\theta(t, x)$ предположить известной, то система (\ref{formula1.1}) станет системой типа $E$(3, 2, 2, 2) и примет вид
    
    \begin{equation}\label{formula1.2}
      \left.
      \begin{array}{rl}
        u_t+uu_x+vu_y+\theta  =  u_{yy}, \\
        u_x+v_y=0.\\
      \end{array}
      \right\}
    \end{equation}

    В отличие от системы (\ref{formula1.1}), здесь функция $\theta$ может рассматриваться в качестве произвольного элемента, в то время как система (\ref{formula1.1}) произвольных элементов не содержит. Поэтому для этих систем уравнений задачи группового анализа различны: для системы (\ref{formula1.1}) это просто задача вычисления ее основной группы (основной алгебры Ли операторов), а для системы (\ref{formula1.2}) — задача групповой классификации по отношению к произвольному элементу $\theta: (t, x) \rightarrow \theta(t, x)$. Дальнейшее изложение и посвящено решению этих задач.
    
      \razd{Предварительная информация об операторе.} Для координат искомого оператора здесь целесообразно применить индивидуальные обозначения, в которых координата обозначается той же буквой, что и скалярная переменная, но со звездочкой наверху.
\end{document}